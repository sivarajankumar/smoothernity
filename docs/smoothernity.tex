\documentclass[12pt]{article}

\usepackage[utf8]{inputenc}
\usepackage{caption}
\usepackage{mathtools}
\usepackage{amssymb}
\usepackage{graphicx}
\usepackage{tikz}
\usepackage{hyperref}
\usepackage{cleveref}
\usepackage{fullpage}
\usepackage{listings}

\usetikzlibrary{arrows}
\usetikzlibrary{calc}
\usetikzlibrary{positioning}
\usetikzlibrary{shapes}

\newenvironment{Figure}
  {\par\medskip\noindent\minipage{\linewidth}}
  {\endminipage\par\medskip}

\crefformat{footnote}{#2\footnotemark[#1]#3}

\hypersetup{colorlinks=true,linkcolor=blue,urlcolor=blue}

\lstset{frame=single,basicstyle=\ttfamily}

\newcommand{\mylstinline}[2]{\fbox{\lstinline[language=#1]{#2}}}

\title{Smoothernity}
\date{\today}
\begin{document}

\maketitle

\tableofcontents

\pagebreak

\section{Idea}

The idea is to create a program which sports the following features:

\begin{enumerate}
    \item
        Huge immersive virtual universe\footnote{
            E.g. from the scale of stars down to planets, buildings and rooms.}.
    \item
        Low physical storage space consumption\footnote{
            E.g. space on hard drive(s) of local or remote machine(s).
            Essentially, this means low \emph{Kolmogorov's comlpexity}
            of the program.}.
    \item
        Real-time universe generation\footnote{
            The world seamlessly generates on-the-fly as player moves through
            the universe.}.
    \item
        Constant frame rate\footnote{
            Smooth visual animation without interruptions during the whole
            time the program is running.}.
    \item
        Ports for different platforms\footnote{
            Both desktop and mobile.}.
    \item
        Effective utilization of the available resources on each
        platform\footnote{
            That is, it should yield ``better picture'' on
            ``faster platforms''.}.
    \item
        Interactive universe editor\footnote{
            The designer can navigate through the universe, make changes
            and see the result as soon as possible.}.
    \item
        Small platform-dependent code part\footnote{
            Let's say, \(\le 10,000\) lines of code for each port.
            E.g. it's OK to have many ports with large \emph{cumulative}
            size as long as none of them exceeds this limit when counted
            separately.}.
    \item
        Peer-to-peer multiplayer.
\end{enumerate}

\section{Generation\label{Gen}}

Whole generation process can be split into several levels of abstraction.

\subsection{Primitives Level\label{Prim}}

Generates reusable low-level building blocks.
Implemented in code, optimized for each platform.
Heaviest computations should be performed on this level.
The examples of primitives are: ``marching cubes of function \(f(x,y,z)\)'',
``perlin noise landscape'', ``spline surface'', etc.
It operates with such entities as ``triangles'', ``meshes'', ``lights'',
``shaders'', ``OpenCL kernels'', ``collision objects'', ``rigid bodies'',
``sound waves data'', etc.

\subsection{Concepts Level\label{Concept}}

Generates reusable high-level building blocks from other reusable building
blocks (both high- and low-level).
Implemented in platform-independent code.
It composes primitives and other concepts into higher-level structures,
such as ``character'', ``vehicle'', ``building'', ``city'', etc.

\subsection{World Level\label{World}}

Generates unique content from high-level building blocks.
Implemented as data.
Connects concepts into directed acyclic graph of dependencies and specifies
attribute values\footnote{
    E.g. city of 20 buildings with 3\ldots5 rooms each.}.
This graph can be edited either interactively by the designer or
programmatically through ``game script''\footnote{
    E.g. add city \(C\) to planet \(P\) after the player had picked up
    item \(I\).}.

Concepts implementation code is itself a dependency for every
instance of this concept in the dependency graph.
Thus, when concept code is changed, all the affected instances and their
dependents can be regenerated.
Implementation-wise, one possible way to do this may be by tracking
all primitives created by the particular concept, so that when this
concept is changed, all primitives can be disposed automatically.

This reloading idea in general case is applicable for static objects only.
That is, object state must depend only on other objects' states, but
not on the self state in a previous moment in time.
It won't work for dynamic case, e.g. because relative phase between
objects' states is changed due to reload.

For dynamic case, we can edit the initial state with frozen time\footnote{
    ``Frozen time'' means that frame updates still occur, but
    time increments are 0.
    It's necessary to let generation go on, while dynamic processes, like
    physics and AI, are halted.
} at 0-time.
Then start simulation by clicking ``play'' button.
To continue editing, we ``pause'' again, reset to 0-time.

Game scenario state can be represented by a vector with integer components.
This vector is essentially an input for world generation.
Some nodes behave conditionally to the game scenario state.
To check game at various stages it's sufficient to change this vector.
Affected nodes can be regenerated on-the-fly, just as regular dependencies.

\section{Runtime}

Runtime state is the most volatile part of game state.
It's not persistent, it exists only while game runs.
Generation part (\Cref{Gen}) defines a shape of the universe.
Runtime part defines its behavior.

Runtime part depends on generation part\footnote{
    Runtime part can ask, which objects should be placed in given
    bounding box of the universe, what is the geometry of these objects,
    etc.
}.
Every resource generated by generation part, is owned by generation part.
Whenever generation part changes\footnote{
    Because game scenario vector has been changed, or
    because of some editor command.
}, runtime part restarts.
That is, all game entities spawned in runtime part\footnote{
    E.g. mesh instances: characters, projectiles, etc.
} are discarded, memory is cleared and runtime script is restarted from scratch.
After restart, runtime part can quickly retrieve resources that were
already generated before restart\footnote{
    Generated resources are owned by generation part, and therefore are
    not affected by runtime restart.
}.

Runtime part can change game scenario vector, thereby effectively
restarting itself.
This is how game script is implemented.
It should be possible to do this restart seamlessly, within single frame update,
so that player wouldn't notice that it occured at all\footnote{
    Because already generated resources will be retrieved instantaneously
    after restart, without regeneration.
}.

\section{Editing}

World level (\Cref{World}) of generation can be edited
interactively.
Changes propagate through directed acyclic graph of dependencies, so
that only affected parts are regenerated.
For desktop computers, interactive editing can be implemented in two
windows: the game (3D scene) and the editor (GUI).

Edited data is valuable, and must be preserved at all costs.
It's desirable to minimize amount of code that can potentially lead to
data corruption.
One way to do this is to separate the game and the editor.
The game is volatile, it should crash as early as possible to diagnose
malfunctions.
The editor should store its data after modification ASAP, to minimize loss
in case if the editor crashes.

The editor consists of two parts: front-end and back-end.
Front-end is essentially a GUI: JavaScript application running in web
browser, that user interacts with.
Front-end is communicating only with back-end.
Back-end is responsible for data persistance and synchronizing with the game.
Back-end communicates with the game and with front-end.
It's preferable to have back-end as a standalone application\footnote{
    E.g. using Nodejs.
}.
This way data should be less likely to get lost or corrupted.
The communication protocol between the game and editor back-end can be
as simple as sending script code to execute on either part.

\section{Scale}

The game universe should be large.
The challenge is that real-time computations\footnote{
    Such as affine transformations, physics simulation, etc.
} require using of floating point numbers, and these have finite precision.
Only a small portion of the universe is immediately visible to the observer,
though.
Thus, one way to mitigate this challenge is to use ``local'' coordinate system,
``centered'' on the observer\footnote{
    That is, all objects near observer will have coordinates near \(0\),
    providing full floating-point precision for computations.
}.
To represent observer's global position in the universe, another set of
coordinates can be used\footnote{
    These coordinates specify observer's ``offset'' from the center of the
    universe.
    If we use double-precision floating point numbers to encode integer
    global coordinates in meters, this gives us a cube with a side of
    \(\approx 2^{15}\) meters, which is \(\approx 1,000\) times larger than
    Solar system.
}.

Due to the limitations of floating-point precision of Z-buffering, there's an
upper limit of the size of the scene, after which Z-fighting appears.
To mitigate this problem one can render scene in multiple passes:
\emph{levels of detail}\footnote{
    For example, render largest scene first (planets, stars).
    Then render smaller scene (landscape up to the horizon, mountains, clouds).
    Finally, render scene closest to the observer
    (grass, characters, buildings).
}.

To navigate through the universe, one can ``shift'' between levels of detail.
Only levels of detail larger or equal to the one the observer is currently in,
are rendered\footnote{
    E.g. if the observer is on the level of ``planets'' and ``stars'',
    he shouldn't see ``grass'' and ``buildings''.
    Only ``planets'' and ``stars'' should be visible.}.
Observer's movement speed is adjusted according to the current level of
detail\footnote{
    Observer should move faster through the level of detail of planets,
    than he moves through the level of detail of grass.
}.

\section{Grid}

Only a small part of the universe is in the immediate vicinity of the observer
in each moment of time.
To split the universe into parts, one may use a \emph{spatial grid}.
Then, the query to the generation algorithm might look like:
``generate the scene in local coordinates, corresponding to the part of the
universe in a box with one corner in \((x_1, y_1, z_1)\) and another in
\((x_2, y_2, z_2)\)''.
The algorithm can figure out how detailed the scene should be from the size of
the query box.

\section{Multiplayer}

Gameplay should be hardcore, relying on player's skill rather than
farming\footnote{
    Like Mortal Kombat, Cave Story, Side Scroller, etc.}.
This requires immediate reaction to player's actions.
Hot-seat multiplayer\footnote{
    Where all players are playing on the same system, sharing
    the screen.
} fits well with this scheme.
Distributed multiplayer in some way should be implemented as well.
It's desirable to use peer-to-peer communication for multiplayer
to avoid spending resources on servers.
There are two modes of multiplayer interactions: active and passive.

\subsection{Passive Mode}

In passive mode players just see each other's position\footnote{
    Like in Dark Souls: show a ghostly form of other players only.
    This approach saves the neccessity to synchronize game worlds
    between players, and softens the annoyance of remote players
    walking through things.
} and chat with each other\footnote{
    \label{Chat}
    Perhaps it's a good idea is to somehow utilize an established chat
    software/service, like IRC, Closed Circles, Facebook, etc.
    This same chat service can serve as an out-of-game mean of
    communication.
    There's an IRC client library in C called
    \href{http://www.ulduzsoft.com/libircclient/index.html}%
    {libircclient}.
}.
Passive mode should be able to function with any connection quality.

There must be a way for clients to discover each other.
Clients maintain a list of other clients they're connected to,
and communicate with them to discover more clients.
Clients ignore clients whose players are too far away, and there must
be a limit of how many other players can be seen by one player\footnote{
    To avoid crowds.
}.
To start discovery, there must be a persistent clients at key
locations\footnote{
    Like towns, portals, dungeon entrances, etc.
}, who are always online, whose sole purpose is to help discovering other
clients.
Each client also maintains a friend list - clients listed there are
prioritized over other clients in discovery process and appear first,
which lets players pick their social circle.
The consequence is that there's no way to discover a client if
they're not near a persistent client or not in a friend list of the
discovering client.
For identity and IP discovery it might be possible to use a chat
system\cref{Chat}.

\subsection{Active Mode}

In active mode players actually play together as in hot-seat mode.
Every player should see the same state of the game universe.
One way to achieve this is to perform the same deterministic game
state updates on each client: collect all players inputs, then
update game state on every client using these same inputs.
This scheme puts all players in the same advantage position:
updates run as fast as slowest client does.
Game logic is computed redundantly in each client, which effectively
eliminates opportunities to cheat by altering game client\footnote{
    E.g. if some client is altered to deal more damage than others,
    it'll only lead to desynchronization: in one client monster
    will be dead, in other - still alive.}.
Thus, it's desirable to play only with players having a good network connection.
Players should be able to choose with whom they want to play
actively\footnote{
    E.g. by sending and accepting invitations.
}, and to kick out players with slow connection.

\section{Implementation}

One of the requirements is portability and small platform-dependent code size.
This means putting as much code as possible to the platform-independend part.
Here is how these two parts may look like.

\subsection{Platform-Dependent Part}

The examples of tasks handled by this part are:
window creation, render context creation, generating and rendering primitives
(\Cref{Prim}), working with network and filesystem, capturing
user input, etc.

As all of these tasks are low-level, natural choice is to use native low-level
language for the target platform.
As of 2014, current bindings of OpenGL and OpenCL to scripting languages are
somewhat limited and outdated.
So in order to use latest available standards, best bet is to use a native
language.

Possible choices are:

\begin{itemize}
    \item In case of Windows or Linux it's C++.
    \item In case of Mac OS X or iOS it might be a mixture of C++ and
        Objective C.
    \item In case of Android OS it's Java.
\end{itemize}

\subsection{Platform-Independent Part}

The examples of tasks handled by this part are:
concepts generation (\Cref{Concept}),
world generation (\Cref{World}),
handling user input, tracking global coordinates,
interacting with the editor, scripts reloading, etc.

Most of the code will be in this part.
Thus, the language must allow for writing compact programs, and also
be fast enough, so that there's little incentive to rewrite this part to
low-level language.
The language must be highly portable and well-adopted in the industry as well.

Possible choices are:
JavaScript\footnote{
    E.g. V8 implementation, as it sports JIT compilation.
    Although V8 isn't cross-platform, so portability might be an issue.
}, LuaJIT, C\#.

\subsection{Scripts Sandboxing}

It might be possible to do scripts reloading entirely from Lua.
\emph{Host} Lua script will load \emph{guest} scripts and run them in
protected mode.
If any of guest scripts throw error, host script intercepts it,
shows notification, asks user to modify offending script and then
reloads it.
It's entirely possible to sandbox guest scripts from within
host script by using \mylstinline{bash}{load} function with custom
environment.
It's also possible to setup a callback firing every \(N\) instructions
using \mylstinline{bash}{debug.sethook}, which'd allow to break
infinite loops.

Remaining reasons to reload host script are:

\begin{enumerate}
    \item Error in host script.
    \item Out-of-memory.
\end{enumerate}

\section{Assorted Notes}

\begin{enumerate}
    \item
        \href{http://gamedev.stackexchange.com/questions/46424/try-catch-or-%
ifs-for-error-handling-in-c}%
        {Discussion}, whether to use or not exceptions in C++ for game clients.
    \item
        Think of a program execution in terms of possible histories of
        changes in the environment\footnote{
            Environment can be a keyboard, display, memory, hard drive, etc.
        }.
        Validity of a program is a constraint on these possible
        histories\footnote{
            E.g. all possible histories for a game should contain states where
            each combination of keys can be pressed or released at any moment
            of time, while no game resource files are modified, and
            there's enough memory.
        }.
    \item
        For networking purposes, one can use ZeroMQ.
        It's small, flexible, mature, portable and well supported.
        There are bindings for Lua and NodeJs.
    \item
        There's a way in NodeJs to monitor file changes in runtime:
        functions \mylstinline{bash}{fs.watch} and
        \mylstinline{bash}{fs.watchFile}.
    \item
        Memory pools should always grow automatically and then report
        resulting sizes in the log.
        The goal is to avoid growing by adjusting initial sizes, but also
        to avoid crashing when these initial sizes are off the mark.
    \item
        There's a code coverage tool for Lua, called
        \href{https://github.com/keplerproject/luacov}{LuaCov}.
    \item
        To verify if the game scenario is passable, one may use automated bots.
        Ideally, when bot is activated, at any point in the game,
        it controls player to complete the story line.
        Bot perceives current state of the game universe and emulates
        player controls\footnote{
            E.g. keyboard, joystick signals, etc.
        }.
        To allow bot to perceive everything that it needs, it may be a good
        idea to design the game around the concept of bot from the very
        beginning.
        
\end{enumerate}

\end{document}
